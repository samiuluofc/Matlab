\documentclass[a4paper,12pt]{article} %make sure that we are using 12 pt font
\usepackage{fancyhdr}
\usepackage[margin=2.5cm]{geometry}%rounded up from 1.87, just to be safe...
\usepackage{parskip}
%%\usepackage{times} %make sure that the times new roman is used
\usepackage{sectsty}
%%\raggedright



\begin{document}


\pagestyle{fancy}
%This puts your name at the top right, outside the margin
\lhead{CPSC 607: Biological Computation (Winter 2016)\\Project Documentation}
\rhead{Samiul Azam\\30 Apr. 2016}
 \renewcommand{\headrulewidth}{1pt}
\sectionfont{\fontsize{12}{15}\selectfont}
\vspace{2cm}


\section*{\\Phase 03: Feature Selection using Genetic Algorithm (GA)}
............................................................................................................................................

In the command window of Matlab, type ``help $<$\verb|file_name.m|$>$'' to see the header documentation of my program scripts.\\
............................................................................................................................................

\textbf{Note 1:}\\\\
This folder contains following main script. Go to Matlab Command Window and type the target main script (For example $>>$  \verb|selection_GA|) for execution.
\begin{itemize}
	\item \verb|selection_GA.m|   
		\begin{itemize}
		\item Read the main script header for more details.
		\item This program will apply binary chromosome based Genetic Algorithm (GA) for selecting features of each classifier.
		\item It generates a ``txt'' file named \verb|Output.txt| (in the current directory) that contains the information about the best individual for each classifier. It also generates following four ``bmp'' files containing generation graphs.
			\begin{itemize}
				\item \verb|disa_GA.bmp|
				\item \verb|knn_GA.bmp|
				\item \verb|tree_GA.bmp|
				\item \verb|svm_GA.bmp|
			\end{itemize}	
	
		\item Moreover, the four chromosomes of the fittest individuals (from four classifiers) are saved in the \verb|chromosome.mat| file.
		\end{itemize}
\end{itemize}

............................................................................................................................................\\\\
\textbf{Note 2: }\\
This folder contains following function scripts (You don't need to run them separately).
\begin{itemize}
		\item \verb|tree_fitness.m|
		\begin{itemize}
			\item Read the function header for more details. Its a fitness function for GA.
			\item This function is used inside \verb|selection_GA.m|.
		\end{itemize}
		\item \verb|disa_fitness.m|
		\begin{itemize}
			\item Read the function header for more details. Its a fitness function for GA.
			\item This function is used inside \verb|selection_GA.m|.
		\end{itemize}
		\item \verb|knn_fitness.m|
		\begin{itemize}
			\item Read the function header for more details. Its a fitness function for GA.
			\item This function is used inside \verb|selection_GA.m|.
		\end{itemize}
		\item \verb|svm_fitness.m|
		\begin{itemize}
			\item Read the function header for more details. Its a fitness function for GA.
			\item This function is used inside \verb|selection_GA.m|.
		\end{itemize}
		
		\item \verb|my_svmclassify.m|
		\begin{itemize}
			\item Not my major contribution. I slightly modify the built-in Matlab function to generate probability instead of `0' or `1' labels. 
			\item This function is used inside \verb|svm_fitness.m|.
		\end{itemize}
		\item \verb|my_svmdecision.m|
		\begin{itemize}
			\item Not my major contribution. I slightly modify the built-in Matlab function to generate probability instead of `0' or `1' labels. 
			\item This function is used inside \verb|my_svmclassify.m|.
		\end{itemize}
\end{itemize}

............................................................................................................................................\\
In the command window of Matlab, type ``help $<$\verb|file_name.m|$>$'' to see the header documentation of my program script.\\
............................................................................................................................................
\end{document}