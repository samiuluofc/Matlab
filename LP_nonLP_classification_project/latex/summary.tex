\documentclass[a4paper,12pt]{article} %make sure that we are using 12 pt font
\usepackage{fancyhdr}
\usepackage[margin=2.5cm]{geometry}%rounded up from 1.87, just to be safe...
\usepackage{parskip}
%%\usepackage{times} %make sure that the times new roman is used
\usepackage{sectsty}
%%\raggedright



\begin{document}


\pagestyle{fancy}
%This puts your name at the top right, outside the margin
\lhead{CPSC 607: Biological Computation (Winter 2016)\\Project Documentation}
\rhead{Samiul Azam\\30 Apr. 2016}
 \renewcommand{\headrulewidth}{1pt}
\sectionfont{\fontsize{12}{15}\selectfont}
\vspace{2cm}


\section*{\\Project Name: License Plate and Non-LP Image Classification}
............................................................................................................................................

Note 1:
\begin{itemize}
	\item The project is implemented in the Matlab 2012a (full version). The project codes are tested and executed on a workstation having Windows operating system.
	
	\item First, run Matlab 2012a (or upper versions). It should have the image processing, global optimization and statistical-and-machine learning toolboxes. Choose ``Project\_Code" as your current folder in the Matlab. Now navigate through subfolders (total 6 folders for 6 phases) to run specific phase of the project. Each of the project-phase folder contains its own README.pdf file. Go through those README.pdf files for more specific instructions. 
	
	\item It's better to run the experiments sequentially (from phase 01 to phase 06) for better understanding. But its not necessary, because each folder contains pre-generated Matlab data files. So you can test any phase at any time.
	
	\item If you find difficulties to run the project, you can contact me at \\email: \verb|samiul.azam@ucalgary.ca|.
	
  	
\end{itemize}

............................................................................................................................................

\textbf{Note 2:}\\\\
Brief description of the project phases:
\begin{itemize}
	\item \textbf{Phase 01:}
		\begin{itemize}
			\item In this phase, I extract HOG features from all Candidate License Plate (CLP) images. Moreover I consider three different resolutions: original size (1x), half (hx) and double (2x) of the original size CLPs. In the next phases, I use only the extracted features during experiment.	  
		\end{itemize}
	
	\item \textbf{Phase 02:}
		\begin{itemize}
			\item In this phase, I tune all four classifiers (based on one of their sensitive parameter) to maximize their performance. In these experiment, I consider all 360 features of HOG. This part is important before applying genetic algorithm (GA) for feature selection.  
		\end{itemize}
	
	\item \textbf{Phase 03:}
		\begin{itemize}
			\item In this phase, I use binary chromosome based GA for feature selection (applied on all four classifiers separately). The selected features (or HOG bins) are used in the next phases.    
		\end{itemize}
	
	\item \textbf{Phase 04:}
		\begin{itemize}
			\item In this phase, I generate trained models for all four classifiers (using selected features). These models are used in the mixture of experts model of phase 05.    
		\end{itemize}
	\newpage
	\item \textbf{Phase 05:}
		\begin{itemize}
			\item In this phase, I have conducted experiment to adjust the four weights of the mixture model. Initially, I assign weights based on participation and performance ranking. Then I apply both GA and Particle Swarm Optimization (PSO) separately to find the best weight vector. PSO performs well in this search process.    
		\end{itemize}
	\item \textbf{Phase 06:}
		\begin{itemize}
			\item In this final phase, I use the searched weight vector (provided by the PSO in the previous phase) inside the mixture of experts model, and test it with 1x, 2x and hx sizes of CLPs.  					
		\end{itemize}
		 
\end{itemize}

..........................................................THANK YOU............................................................

\end{document}