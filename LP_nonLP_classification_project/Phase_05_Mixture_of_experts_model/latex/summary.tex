\documentclass[a4paper,12pt]{article} %make sure that we are using 12 pt font
\usepackage{fancyhdr}
\usepackage[margin=2.5cm]{geometry}%rounded up from 1.87, just to be safe...
\usepackage{parskip}
%%\usepackage{times} %make sure that the times new roman is used
\usepackage{sectsty}
%%\raggedright



\begin{document}


\pagestyle{fancy}
%This puts your name at the top right, outside the margin
\lhead{CPSC 607: Biological Computation (Winter 2016)\\Project Documentation}
\rhead{Samiul Azam\\30 Apr. 2016}
 \renewcommand{\headrulewidth}{1pt}
\sectionfont{\fontsize{12}{15}\selectfont}
\vspace{2cm}


\section*{\\Phase 05: Mixture of Experts Model}
............................................................................................................................................

In the command window of Matlab, type ``help $<$\verb|file_name.m|$>$'' to see the header documentation of my program scripts.\\
............................................................................................................................................

\textbf{Note 1:}\\\\
This folder contains following main script. Go to Matlab Command Window and type the target main script (For example $>>$  \verb|main_GA|) for execution.
\begin{itemize}
	\item \verb|main_W.m|   
		\begin{itemize}
		\item Read the main script header for more details.
		\item This program will give you the accuracies for different preassigned values of weights (before applying Genetic Algorithm (GA) or Particle Swarm Optimization (PSO)).
		
		\item It generates a ``txt'' file named \verb|W_output.txt| (in the current directory) that contains the output.	
			
		\end{itemize}
	\item \verb|main_GA.m|  
		\begin{itemize}
			\item Read the main script header for more details.
			\item This program will search for optimal weight vector using GA Algorithm. It will call GA function for five times (default value of N = 5).  
			
			\item It generates a ``txt'' file named \verb|GA_output.txt| (in the current directory) that contains the outputs (weight vector and accuracy) for all GA calls. Moreover it also generates generation graphs \\\verb|<i>_ensemble_GA.bmp (where i = 1,2,3,....,N)| for all GA calls.	
		\end{itemize}
		
	\item \verb|main_PSO.m|
	\begin{itemize}
		\item Read the main script header for more details.
		\item This program will search for optimal weight vector using PSO algorithm. It will call PSO function for five times (default value of N = 5).  
			
		\item It generates a ``txt'' file named \verb|PSO_output.txt| (in the current directory) that contains the outputs (weight vector and accuracy) for all PSO calls. Moreover it also generates iteration graphs \\\verb|<i>_ensemble_PSO.bmp (where i = 1,2,3,....,N)| for all PSO calls.	
	\end{itemize}
		
\end{itemize}

............................................................................................................................................
\newpage
............................................................................................................................................
\textbf{Note 2: }\\\\
This folder contains following function scripts (You don't need to run them separately).
\begin{itemize}
	
	\item \verb|my_particle_swarm.m|
	\begin{itemize}
		\item Read the function header for more details.
		\item This function is used inside \verb|main_PSO.m|.
	\end{itemize}
	\item \verb|combine_fitness.m|
	\begin{itemize}
		\item Read the function header for more details.
		\item This function is the fitness function (consider 2-fold cross validation, as well as both hx (half zise) and 2x (double size) resolutions of original CLPs) used inside the \verb|my_particle_swarm.m| and the built-in Matlab GA functions.

	\end{itemize}
	\item \verb|my_ensemble_fitness.m|
	\begin{itemize}
		\item Read the function header for more details.
		\item This function is used inside \verb|combine_fitness.m|.
	\end{itemize}
	\item \verb|my_svmclassify.m|
	\begin{itemize}
		\item Not my major contribution. I slightly modify the built-in matlab function to generate probability instead of `0' or `1' labels. 
		\item This function is used inside \verb|my_ensemble_fitness.m|.
	\end{itemize}
	\item \verb|my_svmdecision.m|
	\begin{itemize}
		\item Not my major contribution. I slightly modify the built-in matlab function to generate probability instead of `0' or `1' labels. 
		\item This function is used inside \verb|my_svmclassify.m|.
	\end{itemize}
	
\end{itemize}
............................................................................................................................................
\textbf{Note 3: }\\\\
The folder ``Experiment\_data" contains some pre-generated graph images and text-files.

............................................................................................................................................\\\\

............................................................................................................................................\\
In the command window of Matlab, type ``help $<$\verb|file_name.m|$>$'' to see the header documentation of my program scripts.\\
............................................................................................................................................


\end{document}